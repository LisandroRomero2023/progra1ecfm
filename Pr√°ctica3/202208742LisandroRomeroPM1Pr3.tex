\input{../../../plantilla.tex}

\encabezados[Programación Matemática 1]{Hugo García}
\titulos[Práctica 2]{Programación Matemática 1}

\begin{document}
	
	\thispagestyle{encabezado}
	\pagestyle{general}
	\onehalfspacing
	
	\begin{center}
		\Huge{\underline{\textbf{Práctica 3}}}
	\end{center}
	
	\section{Ideas}
	
	Ya que necesitamos crear un tablero en donde jugar y que este guarde las posiciones y que estas sean editables en cada turno, es una buena idea que las posiciones se guarden como una lista de listas, y que el tablero utilice el formateo para colocar cada miembro de la lista de listas en sus posiciones correspondientes.
	
	Además necesitamos cierta cantidad de acciones que podamos realizar en el tablero:
	\begin{enumerate}
		\item Cambiar el caracter que una posición tiene.
		
		\item Verificar que posiciones están libres.
		
		\item Realizar jugadas en el tablero.
		
		\item Definir si hay un jugador.
	\end{enumerate}
	
	En la misma vía, también necesitamos definir a los jugadores, necesitando de estos el nombre que tendrán y el caracter que utilizarán. Así como necesitamos una forma de iniciar el proceso para llevar a cabo un juego.
	
	Por todo esto, se vuelve una opción mucho más clara la utilización del enfoque orientado a objetos de Python, para que de esta manera sea mucho más simple la interacción en juego.
	
	\section{Código}
	
	\paragraph{Tablero}
	
	El tablero, lo enunciamos de las líneas 11 a la 116. En esto definimos las siguientes funciones para la clase \texttt{Tablero()}:
	
	\begin{enumerate}
		\item \texttt{\_\_init\_\_(self, nombre)}; en donde creamos los atributos para la clase tablero:
		\begin{itemize}
			\item nombre.
			\item tablero, el cual crea el tablero de juego.
			\item posiciones, donde se guardan los elementos en cada punto del tablero.
		\end{itemize}
		
		\item \texttt{imprimirTablero(self)}; en donde imprimimos el tablero asignando las posiciones a sus valores correspondientes en el tablero.
		
		\item \texttt{setPosicion(self, fila, columna, caracter)}; en donde verificamos si la posición del tablero está vacía y si lo está cambiamos la posición por el caracter proporcionado, si no lo está le pedimos al jugador que elija una posición vacía.
		
		\item \texttt{setPosicionCPU(self, fila, columna, caracter)}; en donde verificamos si la posición del tablero está vacía y si lo está cambiamos el valor de la posición por el caracter proporcionado, si no lo está, la máquina elige otra posición que esté vacía.
		
		\item \texttt{getPosiciones(self)}; obtiene las posiciones.
		
		\item \texttt{posicionesLibres(self)}; obtiene una lista que tiene como elementos listas de coordenadas fila-columna, cada lista dentro de la lista retornada es una posición donde es válido jugar.
		
		\item \texttt{posicionesJugador1(self), posicionesJugador2(self)}; devuelve la lista de las posiciones ocupadas por cada jugador.
		
		\item \texttt{jugada(self, jugador)}; realiza una jugada en el tablero pidiendo una posición en la que jugar inicialmente y remitiendo los valores de entrada hacia la función para settear la posición.
		
		\item \texttt{jugadaRandom(self, jugador)}; pide la lista de listas de posiciones libres, de esa lista elige una de las listas para jugar en esa posición, para luego remitir los valores a la función de setteo de posiciones. 
		
		\item \texttt{jugadaOptima(self, jugador)}; función inconclusa que pretendía dar jugadas no perdedoras.
		
		\item \texttt{ganador(self)}; verifica si en las filas, columnas o diagonales hay 3 caracteres iguales, si los hay aumenta en uno el contador de victoria  y cambia el caracter \texttt{M} como el caracter que está en esas tres posiciones encontradas, devuelve la lista \texttt{[victoria, M]}.
		
	\end{enumerate}
	
	\paragraph{Jugador} Creamos con esto a los jugadors, mediante la clase \texttt{Jugador()}:
	
	\begin{enumerate}
		\item \texttt{\_\_init\_\_(self, nombre)}; en donde creamos los atributos para la clase jugador:
		\begin{itemize}
			\item nombre.
			
			\item caracter que utiliza el jugador.
		\end{itemize}
		
		\item \texttt{getCaracter(self)}; devuelve el caracter del jugador.
		
		\item \texttt{getNombre(self)}; devuelve el nombre del jugador.
	\end{enumerate}
	
	\paragraph{juego1v1}
	
	Creamos una función que crea un contador interno para la victoria y un guardado de caracter que se utiliza para validad cuál de los jugadores ganó y cuál perdió. Luego imprime el tablero e inicia un for de 4 ciclos en el cada ciclo juega el primer jugador y a partir del segundo ciclo se verifica se verifica si hay un ganador en cuyo caso de haberlo se asignan los valores del caracter y el contador de victoria, esto se hace de la misma manera para el segundo jugador dentro del mismos ciclo for.
	
	Luego del for se verifica mediante el contador si hay alguien que haya ganado, si no se sigue pidiendole al jugador uno que ocupe la última casilla; y luego se verifica si alguien ganó en cuyo caso se valúa cuál de los jugadores ganó y se expresa, en caso no haya ganado nadie, se imprime un \textit{empate}.
	
	\paragraph{juego1vCPU} Creamos la función \texttt{juego1vCPU(jugador1, jugador2, tablero, opcion)} El algoritmo de juego funciona exactamente igual que en el caso anterior, con la diferencia que la máquina usa las opciones de jugada definidas como óptimas o random.
	
	La opción que se pide en esta función sirve para diferenciar si el juego es de llenado aleatorio o con una estrategia ganadora.
	
	\paragraph{Desarrollo}
	
	Definimos primero el tablero llamándolo \textit{A}, luego preguntamos cuantos jugadores juegan, si hay 1 solo jugador, se pregunta el nombre del jugador 1 con el caracter 'X'; y se crea el jugador computadora (CPU) con el caracter 'O'. Por último se usa la función con las opciones \texttt{juego1vCPU(jugador1, jugador2, tablero, 2)}
	
	Si se tienen 2 jugadores, se preguntan los nombres y se crean dos jugadores con los nombres proporcionados, dandoles el orden de juego según el orden en que introducen el nombre. Por último se usa la función con las opciones \texttt{juego1v1(jugador1, jugador2, tablero)} para inicializar el juego.
	
	El código se encuentra en el siguiente enlace: \href{https://github.com/LisandroRomero2023/progra1ecfm/blob/630e5ebf0edde4ac5c7f573e0d6dd16cb3cf615f/Pr%C3%A1ctica2/C%C3%B3digo/totitoPOO.py#L3}{código del juego.}
	
\end{document}